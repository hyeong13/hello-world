\documentclass[10pt]{beamer}

\usepackage{fancyhdr}
\usepackage{multicol}
\usepackage{graphicx,float,subfigure,caption}
\graphicspath{{figures/}}
\usepackage{amsmath,amssymb}
\usepackage{ulem}
\usepackage{xmpmulti} % for .png sequence of .gif
\usepackage{animate} % for animation of .png sequence of .gif

\newcommand{\handoutgap}[1]{\invisible<0| handout:1->{#1}} %blank in handout

\newlength\dlf
\newcommand\alignedbox[2]{ %boxed answer that breaks across &s
	% #1 = before alignment
	% #2 = after alignment
	&
	\begingroup
	\settowidth\dlf{$\displaystyle #1$}
	\addtolength\dlf{\fboxsep+\fboxrule}
	\hspace{-\dlf}
	\boxed{#1 #2}
	\endgroup
	}

\newcommand{\eiintertext}[1]{\medskip\par\rlap{#1}\medskip} %intertext for itemize & enumerate environments

\usefonttheme[onlymath]{serif} %math appears in serif font

\setbeameroption{show notes}

%\setlength{\parindent}{4em}
\setlength{\parskip}{10pt}

% math commands
\newcommand{\bu}{\mathbf{u}}

\begin{document}
	
	\title{Git Test}
	\frame{\titlepage}
	

\begin{frame}{First frame}

\begin{minipage}{.5\textwidth}
	\begin{itemize}
		\item line 1
		\item line 2
	\end{itemize}
\end{minipage}
\begin{minipage}{.4\textwidth}
	line 3
\end{minipage}

\end{frame}

%	\begin{frame}{Kinematics}
%		\begin{definition}[Kinematics]
%			\handoutgap{The study of classical mechanics which describes the motion of points, bodies, and systems of bodies while \emph{neglecting} the causes of motion.}
%		\end{definition}
%	\end{frame}
%		\note{\begin{itemize}
%				\item right now, we'll ignore the ``causes of motion'' part, but it'll be necessary later
%			\end{itemize}}
%			
%	\begin{frame}{Scalars and vectors}
%		Consider the following: an object travels \SI{7}{\meter} from its initial position. It first moves \SI{4}{\meter} east, then \SI{3}{\meter} north.
%		\begin{itemize}
%			\pause \item \textbf{Distance}: \handoutgap{length of path} = \SI{4}{\meter} + \SI{3}{\meter} = \SI{7}{\meter}
%			\begin{itemize}
%				\pause \item Scalar: magnitude \emph{only}
%			\end{itemize}
%			\pause \item \textbf{Displacement}: \handoutgap{change in position} = \SI{5}{\meter}, $\arctan\left(\frac{3}{4}\right)$ radians north of east
%			\begin{itemize}
%				\pause \item Vector: magnitude \emph{and} direction
%			\end{itemize}
%		\end{itemize}
%	\end{frame}
%		\note{\begin{itemize}
%				\item we often use scalars and vectors, to assign numerical meaning to physical quantities
%				\item think of magnitude as the \emph{raw numerical value} of some quantity
%			\end{itemize}}
%	
%	\begin{frame}{Examples of vectors and scalars}
%		\begin{columns}
%			\column[]{.5\textwidth}
%				Vectors
%				\begin{itemize}
%					\item \handoutgap{Momentum}
%					\item \handoutgap{Force}
%					\item \handoutgap{Torque}
%					\item \handoutgap{Velocity}
%				\end{itemize}
%			\column[]{.5\textwidth}
%				Scalars
%				\begin{itemize}
%					\item \handoutgap{Power}
%					\item \handoutgap{Mass}
%					\item \handoutgap{Temperature}
%					\item \handoutgap{Speed}
%				\end{itemize}
%		\end{columns}
%	\end{frame}
%		\note{\begin{itemize}
%				\item notice that velocity is a vector, while speed a scalar --- this is the common convention we will use
%			\end{itemize}}
\end{document}
